\documentclass{beamer}


\usepackage[utf8]{inputenc}
\usepackage{amsmath}
\usepackage{amsfonts}
\usepackage{amssymb}
\usepackage{graphicx}
\usepackage{ragged2e}  % `\justifying` text
\usepackage{booktabs}  % Tables
\usepackage{tabularx}
\usepackage{tikz}      % Diagrams
\usetikzlibrary{calc, shapes, backgrounds}
\usepackage{amsmath}
\usepackage{amssymb}
\usepackage{listings}  % Code listings
\usepackage[T1]{fontenc}
\usepackage{url}       % `\url
\usepackage{dsfont}
\usepackage{hyperref}

\usepackage{xcolor} %https://en.wikibooks.org/wiki/LaTeX/Colors

\usepackage{theme/beamerthemehbrs}

\author[Kramer]{Erick Kramer}
\title{Introduction to service and client ROS}
\subtitle{}
\institute[HBRS]{Hochschule Bonn-Rhein-Sieg}
\date{04.04.2018}
\subject{Test beamer}

% \thirdpartylogo{path/to/your/image}


\begin{document}
	{
		\begin{frame}
			\titlepage
		\end{frame}
	}
	
	\AtBeginSection[]{% Print an outline at the beginning of sections
		\begin{frame}<beamer>
			\tableofcontents[currentsection]
		\end{frame}%
	}%
	
	\section{Services (Theory)}
	\begin{frame}
		\frametitle{What is a Service?}
		\begin{itemize}
			\item{It is a request/reply communication paradigm.}
			\item{It is composed of two pairs of messages, one for the request and one for the reply.}
			\item{Two nodes participate in the communication process.}
			\item{One node offers the service and other node requests the service and waits for the response.}
			\item{A service is defined using \textbf{srv} files}
			
		\end{itemize}
	\end{frame}
	\begin{frame}
		\frametitle{Service type}
		\begin{itemize}
			\item{The service type is the package resource name of the .srv file, i.e. package name + name of the \texttt{.srv} file.}
			\item{E.g. \colorbox{lightgray}{\texttt{mysrvs/srv/PolledImage.srv}} has the service type \colorbox{lightgray}{\texttt{mysrvs/PolledImage}} }
		\end{itemize}
	\end{frame}
	\begin{frame}
		\frametitle{Service Tools}
		\begin{itemize}
			\item{rossrv - Displays information about .srv data structures.}
			\begin{itemize}
				\item{\colorbox{lightgray}{rossrv show} - Show service description}
				\item{\colorbox{lightgray}{rossrv info} - Alias for rossrv show}
				\item{\colorbox{lightgray}{rossrv list} - List all services}
				\item{\colorbox{lightgray}{rossrv md5} - Display service md5sum}
				\item{\colorbox{lightgray}{rossrv package} - List services in a package}
				\item{\colorbox{lightgray}{rossrv packages} - List packages that contain services}
				
			\end{itemize}
			\item{rosservice - Lists and queries ROS services}
			\begin{itemize}
				\item{\colorbox{lightgray}{rosservice list} - print information about active services}
				\item{\colorbox{lightgray}{rosservice call} - call the service with the provided args}
				\item{\colorbox{lightgray}{rosservice type} - print service type}
				\item{\colorbox{lightgray}{rosservice find} - find services by service type}
			\end{itemize}
		\end{itemize}
	\end{frame}
	\begin{frame}
		\frametitle{Service file}
		\begin{itemize}
			\item{The service file present the following structure: \\
				  \begin{center}
				  	\texttt{\#Request\\
				  message\_type message\\
				  - - -\\
				  \#Response\\
				  message\_type message}
				  \end{center}}
			\item{E.g. \\\begin{center}
				\texttt{
						string str\\
					int8 a\\
					uint32 b\\
					int64 c\\
					- - -\\
					float64 result}
					\end{center}
				}
		\end{itemize}
	\end{frame}
	\section{Services (Practice)}
	\begin{frame}
		\frametitle{Instructions}
		\begin{enumerate}
			\item{Go to the following file \colorbox{lightgray}{\texttt{content/ros/patterns.md}} and follow the instructions.}
		\end{enumerate}
	\end{frame}
	\section{Actions (Theory)}
	\begin{frame}
		\frametitle{Actionlib}
		\begin{itemize}
			\item{Actionlib stack gives a standardized interface containing preemptable tasks, i.e. tasks capable of being interrupted with the option of resuming the task at a later time.}
			\item{Actionlib package allows us to create servers that could perform "long-running" goals.}
			\item{Actions give the ability to cancel a service request during execution.}
			\item{Actions are also useful to get periodic feedback about the request.}
		\end{itemize}
	\end{frame}
	
	\begin{frame}
		\frametitle{Client-Server Interaction}
		\begin{itemize}
			\item{The communication is composed of two elements: ActionClient and ActionServer.}
			\item{The client allows the users to request "goals", while the server executes those goals.}
			\begin{figure}[h!]
				\includegraphics[scale=0.4]{client_server_interaction}
			\end{figure}
			
		\end{itemize}
	\end{frame}
	
	\begin{frame}
		\frametitle{Message specification}
		\begin{itemize}
			%\item{Using an action specification, we need to define Goal, Feedback, and Result messages for the communication between clients and servers.}
			\item{\textbf{Goal: } Provides the sense of accomplishment for a certain task. It is sent to the ActionServer by the ActionClient.}
			\item[-]{ \textit{E.g. for a moving base, a goal could be a "PoseStamped" message that contains the information about where the robot should move.}}
			\item{\textbf{Feedback: } Allows the ActionServer to provides information about the current status of a certain goal to the ActionClient.}
			\item[-]{\textit{E.g for a moving base, the current pose of the robot.}}
			\item{\textbf{Result: } Message sent from the ActionServer to the ActionClient when the goal is completed.}
			\item[-]{\textit{E.g. the final pose of our moving base.}}
		\end{itemize}
	\end{frame}
	\defverbatim[colored]\sleepSort{
		\begin{lstlisting}[language=Python,tabsize=2]
		#Define the goal
		uint32 dishwasher_id
		- - -
		#Define the result
		uint32 total_dishes_cleaned
		- - -
		#Define a feedback message
		float32 percent_complete
		\end{lstlisting}}
	\begin{frame}
		\frametitle{Messages specification}
		\begin{itemize}
			\item{The elements of the actions are specified in the \textbf{.action} file.}
			\item{The action files are placed in a \textbf{./action} directory inside the package.}
			\item{An example of the structure would be}
			\sleepSort
		\end{itemize}
	\end{frame}
	\section{Actions (Example)}
	\begin{frame}
		\begin{itemize}
			\item{\url{http://wiki.ros.org/actionlib_tutorials/Tutorials/Writing\%20a\%20Simple\%20Action\%20Client\%20\%28Python\%29}}
			\item{\url{http://wiki.ros.org/actionlib_tutorials/Tutorials/SimpleActionServer\%28ExecuteCallbackMethod\%29}}
			\item{\url{http://wiki.ros.org/actionlib_tutorials/Tutorials/Writing\%20a\%20Simple\%20Action\%20Server\%20using\%20the\%20Execute\%20Callback\%20\%28Python\%29}}
		\end{itemize}
	\end{frame}
	\section{Reference}
	\begin{frame}
		\frametitle{References}
		\begin{itemize}
			\item{\url{http://wiki.ros.org/srv}}
			\item{\url{http://wiki.ros.org/actionlib}}
		
		\end{itemize}
	\end{frame}
\end{document}
