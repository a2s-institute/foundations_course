\documentclass{beamer}


\usepackage[utf8]{inputenc}
\usepackage{amsmath}
\usepackage{amsfonts}
\usepackage{amssymb}
\usepackage{graphicx}
\usepackage{ragged2e}  % `\justifying` text
\usepackage{booktabs}  % Tables
\usepackage{tabularx}
\usepackage{tikz}      % Diagrams
\usetikzlibrary{calc, shapes, backgrounds}
\usepackage{amsmath}
\usepackage{amssymb}
\usepackage{dsfont}
\usepackage{url}       % `\url
\usepackage{listings}  % Code listings
\usepackage[T1]{fontenc}
\usepackage[percent]{overpic}

\usepackage{theme/beamerthemehbrs}

\author[MAS]{Hassan Umari}
\title{Introduction to ROS}
\subtitle{Foundation Course}
\institute[HBRS]{Hochschule Bonn-Rhein-Sieg}
\date{\today}
\subject{ROS workshop}

% \thirdpartylogo{path/to/your/image}


\begin{document}
{
\begin{frame}
\titlepage
\end{frame}
}


\section{What is ROS?}


\subsection{What ROS is}
\begin{frame}{What ROS is}
\framesubtitle{Robot Operating System}
    \begin{itemize}
        \item Short for: Robot Operating System.
        \item A collection of libraries and tools.
        \item It helps software developers create robot
        applications. 
    \end{itemize}
\end{frame}



\begin{frame}[plain]{}
    \centering
    \includegraphics[width=.66\linewidth]{figures/stop_reinventing_theWheel.jpg}
    
\tiny{http://www.willowgarage.com/blog/2010/04/27/reinventing-wheel}
\end{frame}


\begin{frame}{What ROS is}
    \framesubtitle{Robot Operating System}
    \begin{itemize}
        \item A way to standardize writing software for robots.
        \item It enhances {\huge code reusability} \includegraphics[scale=0.02]{figures/recycling.png}.
        \item ROS is open-source \includegraphics[scale=0.09]{figures/open_source.png}.
        \item It is a meta-operating system.
        \item ROS can be installed on Ubuntu and Debian (so it’s currently supported on Linux only).
        
        \includegraphics[width=.1\linewidth]{figures/linux_logo.png}
        \includegraphics[width=.1\linewidth]{figures/ubuntu_logo.png} 
        \includegraphics[width=.06\linewidth]{figures/debian_logo.png} 
    \end{itemize}
\end{frame}


\subsection{What ROS is NOT}
\begin{frame}{What ROS is NOT}
    \framesubtitle{Robot Operating System}
    \begin{itemize}
        \item It is NOT a programming language.
        \item It is NOT an integrated development environment
        (IDE).
        \item It is NOT a stand-alone operating system 
    \end{itemize}
\end{frame}


%---------------------------

\section{Analogy Between ROS and Operating Systems}

\begin{frame}{Analogy Between ROS and Operating Systems}
    \begin{overpic}[width=.45\linewidth]{figures/analogy1.png}
        \put (130,80) {Software Applications}
        \put (130,50) {work on}
        \put (130,20) {Different hardware}        
    \end{overpic}
\end{frame}

\begin{frame}{Analogy Between ROS and Operating Systems}
    \centering
    \begin{overpic}[width=.4\linewidth]{figures/analogy2.png}
        \put (35,90) {Application} 
        \put (9,65) {\footnotesize Operating System} 
        \put (12,43) {\footnotesize Device Drivers}   
    \end{overpic}
\end{frame}

\begin{frame}{Analogy Between ROS and Operating Systems}
    \centering
    \begin{overpic}[width=.3\linewidth]{figures/analogy3.png}
        \put (2,102) {Mapping algorithm}
        \put (21.5,74) {ROS} 
        \put (11,49) {Laser driver} 
        \put (-25,10) {Laser scanner} 
    \end{overpic}
\end{frame}


\begin{frame}{Analogy Between ROS and Operating Systems}
    \centering
    \begin{overpic}[width=.8\linewidth]{figures/analogy4.png}
    \end{overpic}
\end{frame}





\end{document}
