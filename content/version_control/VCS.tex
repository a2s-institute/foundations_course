
\documentclass{beamer}

\usetheme{default}

\title{Version Control Systems(VCS)}

\author{Kiran Vasudev\inst{1}}

\institute[Universities of Somewhere and Elsewhere] 
{
  \inst{1}
  Hochschule Bonn-Rhein-Sieg

}

\begin{document}

\begin{frame}
  \titlepage
\end{frame}

\begin{frame}{Outline}
  \tableofcontents
\end{frame}

\begin{frame}{What is it?}
\section{What is it?}
  \begin{itemize}
  \item {
    A method to store data/files in a central location.
  }
  \item {
    A backup when things go wrong.
  }
  \end{itemize}
\end{frame}

\begin{frame}{Why should we use it ?}
\section{Why should we use it ?}
  \begin{itemize}
  \item {
    A complete long-term change history of every file.
  }
  \item {   
    Branching and merging
  }
  \item {   
    Ability to trace each change made
  }
  \item {   
	Allows multiple developers to work simultaneously
  }
  \end{itemize}
\end{frame}

\begin{frame}{Types of VCS}{Centralized VCS}
\section{Types of VCS}
\subsection{Centralized VCS}
  \begin{columns}
\column{0.5\textwidth}
\begin{itemize}
  \item {
    Centralized VCS keeps the history of changes on a central server.
  }
  \item {
    Designed with the intent that there is only one true source  
  }
  \item {
    If you want to make a copy of your data, you have to copy/paste it, literally.  
  }
  \item{
	Sourceforge.net uses this type of versioning.  
  }
  \item{
      \textbf{Example:} SVN (Subversion)
  }  
  \end{itemize}
\column{0.5\textwidth}
\begin{figure}
	 \includegraphics[width=.9\textwidth]{images/cvcs}
	 \caption{Centralized Version System \cite{website}}
 \end{figure}
\end{columns}
\end{frame}	

\begin{frame}{Types of VCS}{Decentralized VCS}
\subsection{Distributed VCS}
\begin{columns}
\column{0.5\textwidth}
\begin{itemize}
		  \item {
		    In distributed VCS, everyone has a local copy of the entire work’s history
		  }
		  \item {
		    Each repository is as good as the other
		  }
		  \item{
			Prone to central server crash
		  }
		  \item{
			Your local copy is a repository, and you can commit to it and get all benefits of source control - Offline Source Control.
		  }	  
		  \item{
			Mozilla Firefox uses this kind of versioning.  
		  }
		  \item{
           \textbf{ Example: }Git		  
		  }
	  \end{itemize}
\column{0.5\textwidth}
\begin{figure}
	 \includegraphics[width=.9\textwidth]{images/dvcs}
	 \caption{Distributed Version System \cite{website}}
 \end{figure}
\end{columns}
\end{frame}	

\begin{frame}{Types of VCS}{Git vs SVN}
\subsection{Git vs SVN}

\textbf{Pros of using SVN}
\begin{itemize}
    \item Subversion has better GUIs than Git
    \item Working through versions in Git is tougher than SVN. Git uses SHA-1 hashes while SVN uses sequential revision numbers.    
\end{itemize}
\textbf{Pros of using Git}
\begin{itemize}
    \item Git is much faster than SVN
    \item Git repository file formats are simple and hence  repair and corruption is rare. Space requirements are small
\end{itemize}
\end{frame}

\begin{frame}{Take home message}
\subsection{Take home message}
\begin{itemize}
    \item There is not much difference between the most popular VCSs. It ultimately comes down to the use case.
    \item If there is a need for "Offline Source Control", Git's fantastic. 
    \item If there is a need to have a strictly centralized Source Control that is simple and has excellent tooling (at least on Windows), choose SVN.
\end{itemize}
    
\end{frame}

\begin{frame}{References}
  \section{References}
    
  \begin{thebibliography}{10}

  \bibitem{website}
	\href{https://www.appfusions.com/display/StashSCMImporter/CVCS+vs.+DVCS+In+a+Nutshell
}{CVCS and DVCS}

  \end{thebibliography}
\end{frame}

\end{document}


