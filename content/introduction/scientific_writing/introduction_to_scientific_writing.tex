\documentclass{beamer}


\usepackage[utf8]{inputenc}
\usepackage{amsmath}
\usepackage{amsfonts}
\usepackage{amssymb}
\usepackage{graphicx}
\usepackage{ragged2e}  % `\justifying` text
\usepackage{booktabs}  % Tables
\usepackage{tabularx}
\usepackage{tikz}      % Diagrams
\usetikzlibrary{calc, shapes, backgrounds}
\usepackage{amsmath}
\usepackage{amssymb}
\usepackage{dsfont}
\usepackage{url}       % `\url
\usepackage{listings}  % Code listings
\usepackage[T1]{fontenc}


\usepackage{theme/beamerthemehbrs}
%\usepackage{preamble}


\author[Alex Mitrevski]{Alex Mitrevski}
\title{Introduction to Scientific Writing}
\subtitle{}
%\logo{}
\institute[HBRS]{Hochschule Bonn-Rhein-Sieg}
\date{March 20, 2018}
\subject{Introduction to Scientific Writing}

\begin{document}
{
\begin{frame}
\titlepage
\end{frame}
}

\AtBeginSection[]{% Print an outline at the beginning of sections
\begin{frame}<beamer>
    % \frametitle{Outline for Section \thesection}
    \tableofcontents[currentsection]
\end{frame}%
}%

\section{What is Scientific Writing and Why Should You Care About It?}

\begin{frame}
\frametitle{What is Scientific Writing?}
    Scientific writing is the process of describing scientific ideas and experimental results in a written format
    \newline

    \textbf{Scientific writing is technical writing}; the language used in scientific texts should thus be precise and to the point
    \newline

    Due to its technical nature, which puts several constraints on how and what to write, it is not entirely trivial to do scientific writing right
    \newline

    Scientific texts are usually written for a smaller technical audience
\end{frame}

\begin{frame}
\frametitle{Who Does Scientific Writing?}
    \begin{itemize}
        \item \emph{Researchers}: proposals, conference papers, journal papers, progress reports...
        \item \emph{Students in the sciences}: even if it is just writing up assignments and small project reports
        \item \emph{Scientific journalists}
    \end{itemize}
\end{frame}

\begin{frame}
\frametitle{Why Should You Care About Scientific Writing?}
    \vspace{-1cm}
    As part of your studies, you will have to at least write an R\&D report and a master's thesis; \textbf{both written reports contribute heavily towards your grades}
    \newline

    Those of you who want an academic career will have to continue writing way beyond your studies
    \newline

    Becoming good at writing scientific texts is thus important both for successful completion of your studies and for your future careers
\end{frame}

\section{Types of Scientific Texts}

\begin{frame}
\frametitle{Types of Scientific Texts}
    \begin{itemize}
        \item Project proposals
        \item Project reports
        \item Theses/dissertations
        \item Conference/workshop publications
        \item Journal publications
        \item Other types of scientific texts
    \end{itemize}
\end{frame}

\begin{frame}
\frametitle{Project Proposals}
    \vspace{-1cm}
    A project proposal describes the plan for conducting a project in significant detail
    \newline

    Just as project proposals in other disciplines, scientific project proposals have to convince the reader that the proposer is qualified enough to complete the project
    \newline

    The project's \textbf{feasibility} and \textbf{relevance} are however particularly important in scientific proposals
    \newline

    \textbf{You will have to write proposals for both your R\&D project and your master's thesis}
\end{frame}

\begin{frame}
\frametitle{Project Reports}
    \vspace{-0.5cm}
    A project report summarises the work done in the context of a certain project
    \newline

    Depending on the project's nature, the report's focus might need to be more theoretical or more experimental
    \newline

    Regardless of the nature, such a report should:
    \begin{itemize}
        \item demonstrate that you have learned enough about your project
        \item fully describe your results and any relevant findings
        \item show that your work goes in line with the expectations specified in the proposal or makes it clear why those expectations could not be achieved
        \newline
    \end{itemize}

    \textbf{You will need to write at least one report during your studies: an R\&D project report}
\end{frame}

\begin{frame}
\frametitle{Theses/Dissertations}
    Theses and dissertations are special types of project reports that are written in the context of a master's thesis and a PhD respectively
    \newline

    The scientific aspects of the work are particularly important in theses and dissertations (e.g. the relation of the work to the existing literature and any improvements over the existing work)
    \newline

    In principle, theses and dissertations are longer than a conventional project report
\end{frame}

\begin{frame}
\frametitle{Conference/Workshop Publications}
    \vspace{-1cm}
    A conference publication describes the results of some work in a short and concise way (e.g. ICRA and IROS - six to eight pages)
    \newline

    Such publications give an overview of the work and are excellent for \textbf{communicating ideas to fellow researchers}
    \newline

    Conference/workshop publications are always accompanied by formal gatherings and oral presentations and are thus great for \textbf{networking}
    \newline

    Not all conferences and workshops are of equal quality!
    \newline

    Publishing at high-quality conferences can be seen as a demonstration of one's research worth
\end{frame}

\begin{frame}
\frametitle{Journal Publications}
    \vspace{-1cm}
    Journal publications serve a similar purpose as conference publications, but are generally longer
    \newline

    Journal papers are usually more formal and considerably more detailed than conference/workshop papers
    \newline

    Just as for conferences and workshops, not all journals are of equal quality!
    \newline

    Publishing in high-quality journals is also seen as a demonstration of one's research worth (in some disciplines even more so than publishing at conferences)
\end{frame}

\begin{frame}
\frametitle{Others}
    \vspace{-1cm}
    \begin{itemize}
        \item Handbooks
        \item Textbooks
        \item Lecture notes
        \item Popular science books
        \item Scientific blog posts
        \item Public statements and informal newspaper articles
    \end{itemize}
\end{frame}

\section{Writing Tips}

\begin{frame}
\frametitle{Master the Language in Which You Are Writing}
    \vspace{-1cm}
    To write effectively, you need to have \textbf{excellent} command of the language in which you are writing
    \newline

    Language mastery doesn't just mean being able to use the language on a daily basis, but also knowing its \textbf{grammar} and \textbf{correct practical use}
    \newline

    English is the main language not only in MAS, but also generally in the scientific community, so you should try to master this language as much as possible
\end{frame}

\begin{frame}
\frametitle{Master the Topic You Are Writing About (As Much As Possible)}
    \vspace{-1cm}
    Knowing the language does not help if you don't know what to write about
    \newline

    Scientific writing communicates ideas and results; if they are to be taken seriously, you need to convince your reader that your know what you are talking about
    \newline

    Complete mastery of a topic is impossible and writing often helps you discover the limitations of your knowledge
\end{frame}

\begin{frame}
\frametitle{Write For Your Intended Audience}
    Whenever you write something, you need to have some audience in mind; otherwise, your content is likely to confuse and/or annoy your readers
    \newline

    The audience usually changes depending on the type of manuscript you are preparing
\end{frame}

\begin{frame}
\frametitle{Tell a Story}
    Scientific writing is not a storytelling discipline, but you still need to have a clear "story" in mind while writing
    \newline

    A manuscript that does not seem to have a clear logical flow is likely to be incoherent and thus difficult to read
\end{frame}

\begin{frame}
\frametitle{Read Well-Written Texts and Use Them as Writing Samples}
    \vspace{-1cm}
    While reading papers about your area of interest, you will certainly find some that are very well-written
    \newline

    Use such papers as writing samples and try to learn from them as much as possible
    \newline

    You can however also learn from badly-written papers, i.e. you can try to avoid the things that decrease their writing quality in the first place
\end{frame}

\begin{frame}
\frametitle{Write Outlines}
    \vspace{-1cm}
    Whenever you are faced with a writing task, start by writing an outline
    \newline

    Having an outline helps in multiple ways:
    \begin{itemize}
        \item it gives you an overview of your written work even before you start working on it
        \item you can easily talk about the outline with others (e.g. your supervisors)
        \item it helps you track your writing progress
    \end{itemize}
\end{frame}

\begin{frame}
\frametitle{Let Others Read What You Write}
    \vspace{-1cm}
    While writing, it is very easy to forget that your readers are reading your manuscript, not your mind
    \newline

    Sending manuscript drafts to your supervisors/colleagues/friends is usually very helpful to see what others think about your writing
    \newline

    Not everyone can give you useful comments about your work, so \textbf{send your manuscript to someone who will give you a (brutally) honest opinion about it}
    \newline

    Don't take comments about your writing personally; people are trying to help you improve your work
\end{frame}

\section{Writer's Block and How to Counteract It}

\begin{frame}
\frametitle{What is Writer's Block and Why Does it Happen?}
    \vspace{-0.5cm}
    It sometimes happens that despite the fact that we need to write and despite all our efforts, we just cannot seem to produce any text
    \newline

    This can lead to frustration and procrastination, which will further exacerbate the problem
    \newline

    We call this phenomenon \textbf{writer's block}
    \newline

    Writer's block may have different causes, but stress most likely lies at the core of it\footnote{\url{https://www.ncbi.nlm.nih.gov/pmc/articles/PMC2277565/pdf/canfamphys00047-0094.pdf}}
\end{frame}

\begin{frame}
\frametitle{Tips For Counteracting the Block}
    \vspace{-1cm}
    One very useful technique for counteracting a writer's block is \textbf{free writing}, namely simply writing about what you are supposed to write without thinking too much about the grammar, sentence structure, and wording
    \newline

    Another useful technique is forming/participating in a \textbf{writing/discussion group}
    \newline

    Simply talking with others about your work can also help sometimes
\end{frame}

\begin{frame}
\frametitle{}
    \begin{center}
        {\Huge Thanks for your attention}
    \end{center}
\end{frame}

\end{document}
